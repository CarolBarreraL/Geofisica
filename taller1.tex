%This is a LaTeX template for homework assignments
\documentclass{article}
\usepackage[utf8]{inputenc}
\usepackage{amsmath}
\usepackage{graphicx}

\begin{document}

\section*{Taller 1 - Gravimetr\'ia}
Nombre: Carol Vanessa Barrera L\'opez 
\\ C\'odigo: 201514934
\\Fecha: Septiembre de 2017

\subsection*{1.1 Modelamiento de anomal\'ias gravim\'etricas para sistemas sencillos} 

\begin{enumerate}%starts the numbering

\item {\bf Esfera:} Modelo para los diapiros.
	\begin{enumerate}
	\item Demostraci\'on de la ecuaci\'on:
	blablabla
	\item Se realiz\'o una grafica con tres curvas de la anomal\'ia para diferentes z, cuando la densidad de la esfera era mayor a la densidad del suelo y una curva mostrando la anomal\'ia consecuente a una esfera con una densidad menor que la densidad del suelo (caso del diapiro).
	\includegraphics[width=12cm]{AEsfera.pdf}\centering
	Se puede observar que a mayor z, la curva de la anomal\'ia gravitacional tiene menor altura (por ende menor gmax). Esto se debe a que la relaci\'on entre $g(x)$ y $z$ es directamente proporcional.
	\end{enumerate}

\item Evaluate the following integral 
\begin{equation*}
\int_{2}^{x} \sinh{2y}\, \mathrm{d}y
\end{equation*}
in the case where:
    \begin{enumerate}
    \item $x = 5$
    \\\line(1,0){250}
    \\\line(1,0){250}
    \\\line(1,0){250}
    \item $x = 9$
    \\\line(1,0){250}
    \\\line(1,0){250}
    \\\line(1,0){250}
    \end{enumerate}
    
\item Find the determinant of the following 3x3 matrix:
\begin{figure}[h!]%This example matrix has been enclosed in a figure to give us more positioning options
\centering
\begin{math}
\begin{pmatrix}
1 & 6 & 4 \\
3 & 1 & 7 \\
9 & 4 & 5
\end{pmatrix}
\end{math}
\end{figure}
\\\line(1,0){300}
\\\line(1,0){300}
\\\line(1,0){300}

\end{enumerate}%ends the numbering

\subsection*{Section B}

\begin{enumerate}

\item Determine the following limit:
\begin{equation*}
\lim_{x \to +\infty} \frac{\sin{x}}{x^2}
\end{equation*}
\\\line(1,0){300}
\\\line(1,0){300}
\\\line(1,0){300}

\item Prove Pythagoras' Theorem.
\\\line(1,0){300}
\\\line(1,0){300}
\\\line(1,0){300}

\item Describe the Gram-Schmidt Algorithm.
\\\line(1,0){300}
\\\line(1,0){300}
\\\line(1,0){300}

\end{enumerate}

\end{document}
